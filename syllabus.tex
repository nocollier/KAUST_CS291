\documentclass[12 pt]{article}
\usepackage[margin=1in]{geometry}
\usepackage[colorlinks=true]{hyperref}
\title{CS291: Scientific Software Engineering}
\author{Nathan Collier}
\begin{document}
\maketitle
\paragraph{Prerequisites} Programming experience and familiarity with basic discrete and numerical algorithms.

\paragraph{KAUST course description}

Practical aspects of application development for high performance
computing. Programming language choice; compilers; compiler
usage. Build management using make and other tools. Library
development and usage. Portability and the GNU auto-configure
system. Correctness and performance debugging, performance
analysis. Group development practices and version control. Use of
third party libraries and software licensing.

\paragraph{Alternative course description}

The purpose of this class is to teach the principles of software
development with particular emphasis in the development of scientific
applications. Many computational scientists come from application
areas and develop software as best they can, unaware of many practices
which are common in computer science. We will learn scientific
software development actively by learning to participate in a current
project. We will discuss abstract data structures, programming
language choice, portability, and performance.

\paragraph{Reference Material}

\begin{itemize}
\item Textbooks on software engineering \cite{Sommerville2011,Nanz2011}
\item Opinions on the role of scientific software engineering in academia \cite{Todorov2012,Dirk2012}
\item Thoughts on scientific software \cite{Katzgraber2010,Hannay2009,Smith2007}
\end{itemize}

\paragraph{Rough outline}


\bibliographystyle{plain} \bibliography{course}
\end{document}
